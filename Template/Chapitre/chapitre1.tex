\pagenumbering{arabic}
\setcounter{page}{1}
\fancyhead[l]{}
\nochapter{INTRODUCTION GENERALE}

\vspace{0.4cm}
\hspace{0.5cm}
\section*{Contexte et justifcation}
Le Ghana, avec un PIB de 76,63 milliards USD en 2023, est la deuxième économie de la CEDEAO, derrière le Nigéria et devant la Côte d'Ivoire. L'économie ghanéenne repose principalement sur l'exploitation des matières premières telles que l'or, le pétrole, le gaz . Avant la pandémie de COVID-19, le Ghana bénéficiait d'un taux de croissance d'environ 7\%, mais la croissance a ralenti à 0,5\% en 2020 avant de rebondir à 5,4\% en 2021, et en  2022, l'économie s'est contractée à 3,2\% en raison d'une inflation élevée.

\vspace{0.4cm}
Dans un contexte économique marqué par une inflation record de 54,1\% en 2022, une dépréciation du cédi, un ratio dette publique/PIB de 89\% et une perte de confiance des investisseurs internationaux, le Ghana traverse une crise financière et économique profonde. Les réserves internationales ont diminué rapidement, atteignant moins de 2 semaines d'importations en février 2023. Le gouvernement a eu de plus en plus recours au financement monétaire auprès de la Banque du Ghana, ce qui est désormais impossible dans le cadre du programme du FMI. En mai 2023, le FMI a accordé une Facilité Elargie de Crédit de 3 milliards de dollars au Ghana pour aider à stabiliser l'économie.

\vspace{0.4cm}
\section*{Problématique}
	Malgré cette aide financière, nombreux sont les secteurs, notamment le secteur du cuir  qui font face  à plusieurs défis : la dépendance à l’exportation de matières premières brutes, le manque d’infrastructures de transformation locale, les coûts élevés de production et la concurrence accrue des produits importés. Cependant, le marché du cuir au Ghana est en croissance, avec une augmentation prévue de 4,2\% par an entre 2020 et 2026, soutenue par la demande croissante de produits en cuir tels que les vêtements, les chaussures et les accessoires, tant sur les marchés domestiques qu'internationaux. Il est donc pertinent de s'interroger sur les spécificités  et les caractéristiques du marché du cuir au Ghana pour élaborer des stratégies efficaces et adaptées. Ceci dit, quels sont  donc les opportunités et les perspectives  de développement économique du Ghana à travers le marché du cuir ?
\section*{Objectifs}
	Tout au long de notre analyse, il sera principalement question pour nous d'identifier les opportunités de developpement du marché du cuir au Ghana; Plus specifiquement, il sera question de : 
	\begin{itemize}
		\item Présenter l'économie du cuir aux niveaux mondial et africain ;
		\item Présenter les spécificités du marché du cuir au Ghana;
		\item Identifier les enjeux et les défis liés au marché du cuir au Ghana ;
		\item Formuler les recommandations de développement de la filière du cuir au Ghana
	\end{itemize}
\section*{Méthodologie et outils}
	Les analyses ont été effectuées a partir des données secondaires provenant d'organisations internationales (FAO en l'occurrence), des Think tank et de revues économiques. Elles ont été inscrites dans le cadre théorique des structures de marché de concurrence imparfaite. 
\vspace{0.4cm}
\section*{Plan du travail}
	L'analyse se déclinera en quatre parties :  nous présenterons tout d'abord, l'économie du Ghana, en examinant son rôle sur les marchés internationaux et son panorama économique national ; ensuite, nous ferons une description  sommaire du marché du cuir au Ghana en, en identifiant les zones de production, les régulations et les acteurs majeurs ; puis nous effectuerons une analyse du marché, en synthétisant les études existantes, l'offre et la demande, la structure du marché et la chaîne de valeur ; et enfin nous présenterons  les perspectives de développement et recommandations stratégiques, afin de proposer des solutions concrètes pour le futur.
%________________________________________________
\newpage
\chaptitle
\fancyhead[l]{}
\vspace{0.2cm}\ochapter{PRESENTATION DES DONNEES ET ANALYSES EXPLORATOIRES}\vspace{-0.7cm}


\section{Présentation des données}

\subsection{Source et nature des données}

Les données utilisées dans cette étude proviennent d'une commande \texttt{ping} exécutée pour mesurer la latence d'accès au serveur DNS public de Google (\texttt{8.8.8.8}). À chaque exécution de la commande, des paquets ICMP (Internet Control Message Protocol) sont envoyés à l'adresse cible à un intervalle d'une seconde. Chaque ligne de réponse représente l'état de la requête envoyée et contient des informations sur la latence ainsi que d'autres paramètres relatifs à la requête.

\subsubsection{Informations clés des données}

Chaque ligne contient plusieurs paramètres essentiels, qui fournissent des informations sur l'état de la requête ICMP envoyée ainsi que la latence observée. Les lignes peuvent être classées en plusieurs catégories selon l'état de la requête :

\begin{itemize}
	\item \textbf{Requête aboutie} : Représentée par une ligne contenant le temps de réponse en millisecondes (latence), comme par exemple \texttt{time=XXX ms}.
	\item \textbf{Requête non aboutie} : Certaines requêtes échouent, et la ligne contient un message indiquant l'échec de l'envoi, comme par exemple \texttt{ping: sendto: No route to host}.
	\item \textbf{Requête aboutie sans retour} : Dans certains cas, une requête peut atteindre la destination sans retour de données, résultant en un message de \texttt{Request timeout for icmp\_seq=XXX}, signifiant que la requête a été envoyée mais aucune réponse n'a été reçue dans le délai imparti.
\end{itemize}

\subsubsection{Structure d'une requête reussie (réponse)}

Une ligne représentant une requête réussie contient plusieurs informations structurées, qui permettent de mesurer et d'analyser la latence ainsi que d'autres caractéristiques du réseau. La structure d'une ligne typique de réponse réussie est la suivante :

\begin{itemize}
	\setstretch{1.3}
	\item \textbf{Taille du paquet reçu} : \texttt{64 bytes from 8.8.8.8} (indique que le paquet contient \textbf{64 octets} et provient de l'adresse IP cible).
	\item \textbf{Numéro de séquence ICMP} : \texttt{icmp\_seq=XX} (\textbf{XX} représente un identifiant unique incrémenté à chaque envoi).
	\item \textbf{TTL (Time-To-Live)} : \texttt{ttl=105} (indique le nombre maximal de sauts restants avant que le paquet soit rejeté).
	\item \textbf{Temps de réponse (latence)} : \texttt{time=XXX ms} (le temps \textbf{aller-retour} entre l'ordinateur source et l'adresse cible, exprimé en millisecondes).
\end{itemize}

Ces quatre éléments permettent de définir et d'analyser les caractéristiques d'une requête réussie, comme le temps que le paquet met à voyager entre l'ordinateur source et le serveur cible, ainsi que la fiabilité du réseau (indiquée par le TTL).

La date et l'heure de départ de la commande \texttt{ping} sont enregistrées dans la première ligne de la capture. Dans ce cas précis, la commande a été exécutée le :

\begin{itemize}
	\setstretch{1}
	\item \textbf{Date et heure de départ} : Vendredi 31 janvier à 09:23:12.
	\item \textbf{Adresse cible} : 8.8.8.8 (serveur DNS public de Google).
	\item \textbf{Intervalle de collecte} : Chaque seconde entre chaque requête envoyée.
\end{itemize}

\section{Analyses exploratoires}

\begin{table}[H]	
	\captionsetup{justification=raggedright, singlelinecheck=false}
	\centering
	\caption{Répartition des principales exportations du Ghana}\vspace{-0.2cm}
	\setstretch{0.65}\fontsize{10}{20}\selectfont
	\begin{tabular}{||>{\raggedleft\arraybackslash}m{5.5cm}||c||}
		\hline
		\textbf{Produit} & \textbf{Part des exportations (\%)} \\
		\hline
		Or, y.c. l'or platiné, sous formes brutes ou transformées & 45,4 \\
		\hline
		Huiles brutes de pétrole ou de minéraux bitumineux & 23,6 \\
		\hline
		Cacao en fèves et brisures de fèves, bruts ou torréfiés & 6,6 \\
		\hline
		Pâte de cacao, même dégraissée & 2,4 \\
		\hline
		Minerais de manganèse et leurs concentrés, y.c. & 1,9 \\
		\hline
		Noix de coco, noix du Brésil et noix de cajou, fraîches & 1,5 \\
		\hline
		Beurre, graisse et huile de cacao & 1,4 \\
		\hline
		Préparations et conserves de poissons; caviar et autres & 0,8 \\
		\hline
		Graisses et huiles végétales - y.c. l'huile de palme & 0,7 \\
		\hline
		Barres en fer ou en aciers non-alliés, simplifiées & 0,7 \\
		\hline
	\end{tabular}
	\subcaption*{Source : Fellah Trade, \textit{Carte de l'échange au Ghana}, }
\end{table}
\vspace{-1.5cm}
\subsection{Les importations}
\begin{figure}[H]	
	\centering 
	\caption{ RNB par habitant (en ppa)}
	\includegraphics[width=0.55\textwidth, keepaspectratio]{Images/logo_issea.png}
	\subcaption*{Source : \emph{World Economic Outlook (WEO), FMI.  }}
\end{figure}

%_________________________________________________________________________________________
\newpage
\chaptitle
\fancyhead[l]{}
\vspace{0.2cm}\ochapter{ANALYSES DE LA PERFORMANCE DU RESEAU AVEC PYSPARK}\vspace{0.4cm}
\hspace{0.5cm}
\fancyhead[l]{{\small \textit{Revue de la litterature}}}

%_________________________________________________________________________________________
\newpage
\chaptitle
\fancyhead[l]{}
\vspace{0.2cm}\ochapter{INTERPRETATION ET DISCUSSION DES RESULTATS}\vspace{0.4cm}




%_________________________________________________________________________________________

\newpage 
\fancyhead[l]{}
\pagenumbering{roman}
\renewcommand{\thepage}{\MakeUppercase{\roman{page}}}
\setcounter{page}{5}
\newpage 
\vspace{0.2cm}\nochapter{CONCLUSION ET RECOMMENDATIONS}\vspace{0.4cm}

Le marché du cuir au Ghana présente les caractéristiques d’une concurrence monopolistique, avec une grande diversité de produits et de nombreux petits producteurs locaux, principalement dans le nord et à Kumasi. Cette structure favorise l’innovation et la différenciation, mais pose également des défis, notamment la fragmentation du marché, la variation de la qualité des produits et le manque d’infrastructures modernes.Les barrières à l’entrée restent modérées grâce à un soutien gouvernemental, mais la modernisation des outils et la standardisation des produits sont nécessaires pour renforcer la compétitivité, notamment sur les marchés internationaux. En termes d'efficacité économique, bien que la concurrence monopolistique permette une certaine flexibilité et dynamisme, l'évolution vers un marché moins concentré pourrait améliorer la compétitivité et l'accès à des technologies modernes, tout en maintenant un équilibre entre la production artisanale et l'industrialisation. Cependant, un tel changement devrait prendre en compte la protection des petits producteurs locaux afin de préserver la diversité et l’artisanat, essentiels à l'identité du marché ghanéen du cuir.

%_________________________________________________________________________________________
\newpage
%\nocite{*}
%\vspace{0.2cm}\nochapter{BIBLIOGRAPHIE  ET WEBOGRAPHIE}\vspace{0.4cm}
%\renewcommand{\bibname}{klnoza} 
%\renewcommand{\refname}{}
%\printbibliography[heading=none]% none fait en sorte que cela ne cre pas ue nouvelle page

%________________________________________________________________________

