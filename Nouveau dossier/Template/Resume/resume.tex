\nochapter {RESUME}


\vspace{.5cm}
	 Le Ghana, bien que relativement nouveau dans la production de cuir à grande échelle, joue un rôle clé dans les échanges régionaux grâce à ses zones de production principalement situées dans le nord et à Kumasi, où la production artisanale reste prédominante. Le marché du cuir ghanéen se distingue par une structure de concurrence monopolistique, marquée par une forte fragmentation avec de nombreux petits producteurs, chacun offrant des produits différenciés (chaussures, sacs, accessoires décoratifs, etc.). Ce phénomène favorise la diversité, mais aussi des défis en matière d’harmonisation des produits et de compétitivité internationale. L’étude souligne également les obstacles à la modernisation des techniques de production, les barrières financières et techniques pour les nouveaux entrants, ainsi que les limitations liées au manque d’infrastructures modernes et à la standardisation des produits. 
	 
	 Cependant, le marché bénéficie d’une régulation en développement, ainsi que d’un soutien gouvernemental qui facilite l’acquisition de compétences pour les artisans à travers l’enseignement du travail du cuir dans les écoles. À travers l’analyse de la structure du marché et des dynamiques de l’offre et de la demande, l’étude propose plusieurs recommandations stratégiques pour améliorer l'efficacité du secteur, telles que l'investissement dans l'innovation, l'amélioration des infrastructures et la mise en place d’un cadre favorisant une meilleure organisation de la chaîne de valeur. Cela permettrait non seulement de renforcer la compétitivité du cuir ghanéen sur le marché international, mais aussi de stimuler un développement durable du secteur, tout en soutenant les petites entreprises artisanales locales et leur intégration dans un marché plus vaste.